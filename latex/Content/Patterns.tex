\section{Patterns}
\begin{frame}{Definição de \textit{patterns}}
  \textbf{Patterns = Regex + Magic}
  \begin{widedescription}
      \item \begin{quotation} \small\it
        Some characters in the pattern, such as letters, are taken literally [...]
        Other characters have a special meaning without a backslash. If a character
        is taken literally or not depends on the 'magic' option and the items in the
        pattern mentioned next\cite{vimReferenceManual}.
      \end{quotation}
  \end{widedescription}
\end{frame}

\begin{frame}{Definição de \textit{patterns}}
  \textbf{REGEX}
  \begin{widedescription}
    \item Uma \textbf{expressão regular} pode ser definida como uma linguagem ou sequência de caracteres que é aceita
    por um autômato finito;
    \item \textbf{Expressões Regulares} são usualmente adotadas como uma solução para identificar ocorrências de
    padrões em processamento de texto.
  \end{widedescription}
\end{frame}

\begin{frame}{Uma breve história}
  \begin{widedescription}
    \item O conceito foi formalizado em 1951 pelo matemárico americano Stephen Cole Kleene \cite[kleene1956];
    \item Expressões regulares foram popularizadas da década de 1980 em diante, tornando-se o foco principal de muitas
          ferramentas essenciais de sistemas UNIX, como \texttt{grep}, \texttt{sed}, e \texttt{AWK}.
  \end{widedescription}
\end{frame}

\begin{frame}[fragile]\frametitle{Alguns exemplos de expressões regulares}
    \begin{wideitemize}
      \item \verb|/[A-Z]\w*/|
      \item \verb$/\.[a-z]\+\(config\|rc\)\?\.\(json\|ya\?ml\)/$
    \end{wideitemize}
\end{frame}

\begin{frame}[fragile]\frametitle{Operadores}
  \begin{columns}
    \begin{column}{0.5\textwidth}
      \begin{widedescription}
        \item \verb|^| Início de linha
        \item \verb|$| Final de linha
        \item \verb|*| Zero ou mais vezes
        \item \verb|+| Uma ou mais vezes
      \end{widedescription}
    \end{column}
    \begin{column}{0.5\textwidth}
      \begin{widedescription}
        \item \verb$|$: Operador "OU"
        \item \verb$?$: Opcional
        \item \verb$[]$: Agrupamento de classes
        \item \verb$()$: Agrupamento de expressões
      \end{widedescription}
    \end{column}
  \end{columns}
\end{frame}

\begin{frame}[fragile]\frametitle{Classes de caracteres}
  \begin{widedescription}
    \item \verb$\w$: Word character
    \item \verb$\d$: Digit
    \item \verb$\s$: Whitespace
  \end{widedescription}
\end{frame}
